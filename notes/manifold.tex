
%\documentclass{article}
\documentclass[18pt]{extarticle}
% Comment the following line to NOT allow the usage of umlauts
\usepackage[utf8]{inputenc}
% Uncomment the following line to allow the usage of graphics (.png, .jpg)
%\usepackage{graphicx}

\newcommand*{\pd}[3][]{\ensuremath{\frac{\partial^{#1} #2}{\partial #3}}}
% Start the document
\begin{document}

% Create a new 1st level heading
\section{Finding a function's derivative in chart coordinates}
Front matter
$
\\ f(x) = f(\psi^{-1}(\psi(x)))
%
\\ \frac{\partial}{\partial x} f(x) =
\frac{\partial }{\partial x} 
f(\psi^{-1}(\psi(x)))
%
\\ =\pd{f}{x}(\psi^{-1}(\psi(x))) *
\pd{}{x}(\psi^{-1}(\psi(x))) 
%
\\ =\pd{f}{x}(\psi^{-1}(\psi(x))) *
\pd{\psi^{-1}}{x}(\psi(x)) *
\pd{}{x}\psi(x)
\\ $
\newline
Now also consider the following:
\newline
$
\\ \pd{}{x}f(\psi^{-1}(x)) =
\pd{f}{x}(\psi^{-1}(x)) *
\pd{\psi^{-1}}{x}(x)
$
\newline
\newline
This is the derivative of the function in the chart coordinates on the left hand side and then applying the chain rule on the right.
\newline 
But from the above equalities we see we can relate
the derivative of the function in chart coordinates, to its derivative in manifold coordinates by (a) evaluating it at $\psi(x)$ and then (b) multiplying by $\pd{\psi}{x}$.
\newline 
\newline
So we finally have:
\newline
$
\\ \pd{}{x}f(x) = \pd{\psi}{x}[
\pd{\psi^{-1}}{x}@\psi(x) * 
\pd{}{x}f(\psi^{-1}(x)) @ \psi(x)]
$
\end{document}
